\documentclass[a4paper,12pt]{report}
\usepackage{alltt, fancyvrb, url}
\usepackage{graphicx}
\usepackage[utf8]{inputenc}
\usepackage{float}
\usepackage{hyperref}

\usepackage[italian]{babel}

\usepackage[italian]{cleveref}
\title{Progetto di Programmazione di Reti 
    \\ Traccia 1: Chat Client-Server}

\author{Matteo Giorgini \\ Matricola: 0001136576}
\date{11 giugno 2024}   
\begin{document}
\maketitle
\tableofcontents
\chapter{Consegna}
Implementare un sistema di chat client-server in Python utilizzando socket programming. 
Il server deve essere in grado di gestire più client contemporaneamente e deve consentire agli utenti di inviare e ricevere messaggi in una chatroom condivisa. 
Il client deve consentire agli utenti di connettersi al server, inviare messaggi alla chatroom e ricevere messaggi dagli altri utenti.
\chapter{Server}
Per realizzare il server ho dovuto importare dalla libreria socket la funzione socket per creare un socket e le costanti AF INET, 
per indicare che il protocollo usato è di tipo IPv4, e SOCK STREAM, in modo che il socket sia orientato alla connessione secondo il protocollo TCP.
Le funzionalità richieste sono svolte da thread diversi, utilizzando la libreria importata Threading, i quali utilizzano 2 funzioni:
\begin{itemize}
    \item \texttt{incoming\_connections}: viene attivata all'avvio dell'applicazione e svolge il compito di ascoltare e accettare le richieste di connessione al server da parte di un client attraverso un flag SYN.
    \item \texttt{manage\_client}: gestisce un singolo client e per ogni messaggio ricevuto da un client lo invia a tutti gli altri utilizzando la funzione \texttt{broadcast},
    \\ la quale manda con il nome del client mittente il messaggio a tutti i client memorizzati. Se riceve come messaggio quello di terminazione della connessione ("{quit}"), termina il client e notifica tutti gli altri che si è disconnesso.
\end{itemize}
\chapter{Client}
Per realizzare il client ho utilizzato oltre alle libreria utilizzate per il server anche la libreria tkinter per realizzare una piccola interfaccia grafica dove inviare e ricevere i messaggi della chat. 
Quando un client si vuole connettere viene creato un socket per la connessione al server con la funzione connect che manda il messaggio con il flag SYN. Dopo aver stabilito la connessione viene creato un thread per ricevere i messaggi attraverso la funzione recv. 
I messaggi vengono ricevuti tramite la funzione \texttt{receive\_messages} e vengono mostrati nell'interfaccia grafica insieme a quelli inviati. 
Per inviare messaggi viene utilizzata la funzione \texttt{send\_messages} che invia il messaggio tramite la socket creata sul server. 
Se si invia il messaggio per terminare la connessione viene chiuso il socket e l'interfaccia grafica e viene inviato un messaggio di disconnessione.
\chapter{Funzionamento}
Per utilizzare gli script è necessario installare Python se non è presente sulla macchina. Come prima cosa è necessario eseguire lo script server.py che apre la porta sul server e attende le richieste di connessione.
Successivamente si può eseguire lo script client.py su più terminali in modo da simulare più utenti connessi alla chat. All'avvio del client verrà richiesto di inserire l'indirizzo del server (in questo caso è stato utilizzato 127.0.0.1) e la porta (42069),
se queste informazioni risultano errate viene chiesto di reinserirle. Dopo aver effettuato il collegamento al server si avvia l'interfaccia grafica e come prima cosa bisogna digitare e inviare il proprio nome utente,
dopodichè sarà possibile partecipare alla chat inviando e ricevendo messaggi, ed eventualmente uscire inviando il rispettivo comando {quit};
\end{document}